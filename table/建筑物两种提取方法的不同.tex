% !TeX root = ../main.tex

\begin{longtable}[htbp]{|>{\bfseries}m{0.2\textwidth}|m{0.35\textwidth}|m{0.35\textwidth}|}
	\caption{建筑物两种提取方法的不同}
	\label{tab:建筑物两种提取方法的不同}
	
	\\\hline
	\thead{比较内容}  & \thead{基于影像} & \thead{基于LiDAR} \\\hline
	\endfirsthead
	% 后续页开头
	\multicolumn{3}{r}{\kaishu(续表)} \\
	\hline
	\thead{比较内容}  & \thead{LiDAR} & \thead{INSAR} \\\hline
	\endhead
	% 结尾
	%\hline
	\endfoot
	% 最后页结尾
	%\hline
	\endlastfoot
	
	建筑物的信息不同 
	& 	影像信息具有丰富的语义信息,可以判定建筑物的存在、结构和性质。
		但是城市建筑物的屋顶在影像中通常是纹理频发的区域,且由于投影差的影响,通常难以准确匹配,即使能够匹配,其精度也比较低。
	& 	LiDAR可以直接得到较为密集的屋顶离散点数据,
		容易通过高程将建筑物与同材质的地面区分 \\
	\hline
	提取策略不同
	&	摄影测量方法形成建筑物模型周期长,代价大。
		利用航空立体相对获取建筑物的方法,是通过获取点、线、面特征,再得到三维模型,可以看做是自下而上的处理流程。
	&	利用LiDAR的建筑物模型生成通常采用建筑物拟合生成。
		通过整体的建筑物数据逐步细化每一个屋顶模型,可以看做是自上而下的方法。\\
	\hline
	主要误差源不同
	&	一般影像中建筑物的特征容易受到各种光照情况的干扰,如:阴影、屋顶的涂料颜色等。
	&	LiDAR数据是高程数据,不受光谱信息的干扰,精度较高。其主要问题在于边缘处可能没有信息,造成建筑物边缘的提取误差。\\
	\hline
	处理方法不同
	&	基于影像的建筑物提取和建模方法,经过几十年的发展,已经比较成熟,有较为固定的流程和算法。
	&	基于LiDAR数据的建筑物提取和建模方法还在不断摸索中,还没有很成熟的流程和方法。\\
	\hline
\end{longtable}